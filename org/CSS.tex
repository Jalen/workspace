% Created 2012-01-07 六 21:12
\documentclass[11pt]{article}
\usepackage[utf8]{inputenc}
\usepackage[T1]{fontenc}
\usepackage{graphicx}
\usepackage{longtable}
\usepackage{hyperref}


\title{CSS}
\author{Jalen Wang}
\date{07 一月 2012}

\begin{document}

\maketitle

\setcounter{tocdepth}{3}
\tableofcontents
\vspace*{1cm}

\section{数据和视图分离}
\label{sec-1}

\subsection{有了 CSS,就可以控制页面的外观并且将文档的表现部分与内容分隔开}
\label{sec-1.1}

\section{更易于保持各个界面的风格的一致性}
\label{sec-2}

\subsection{这个界面可以使用相同的CSS}
\label{sec-2.1}

\section{标记}
\label{sec-3}

\subsection{id 和 类名}
\label{sec-3.1}

\subsection{div 和 span}
\label{sec-3.2}

   div 可以用来对块级元素进行分组,而 span 可以用来对行内元素进行分组或标识
\section{为样式找到目标}
\label{sec-4}

  有效且结构良好的文档为应用样式提供了一个框架。要想使用 CSS 将样式应用于特定的(X)HTML 元素,
需要有办法找到这个元素。在 CSS 中,执行这一任务的样式规则部分称为选择器(selector)。
\subsection{常用的选择器}
\label{sec-4.1}

\subsubsection{类型选择器}
\label{sec-4.1.1}

    类型选择器有时候也称为元素选择器或简单选择器

\begin{verbatim}
p { color: black;}
a { text-decoration: underline;}
h1 { fon-weight: bold;}

\end{verbatim}


\subsubsection{后代选择器可用来寻找特定元素或元素组的后代}
\label{sec-4.1.2}

    后代选择器由其他两个选择器之间的空格表示。在下面的示例中,只在作为列表项的后
    代的锚元素上应用样式,而段落中的锚不受影响。

\begin{verbatim}
li a { text-decoration: none;}

\end{verbatim}


\subsubsection{ID 选择器由一个\#字符表示,类选择器由一个点号表示}
\label{sec-4.1.3}


\begin{verbatim}
#intro { font-weight: bold;}
.dateposted { color: green;}
<p id="intro">some text</p>
<p class="dateposted">24/3/2006</p>

\end{verbatim}


\subsubsection{伪类}
\label{sec-4.1.4}


\begin{verbatim}
/*maks all unvisited links blue*/
a: like{ color: blue;}
/*makes all visited links green*/
a: visited { color: green;}
/*maks links red when hovered or activated*/
a: hover, a: active{ color: red;}
/*maks table rows red when hovered over*/
tr: hover { background-color: red;}
/*maks input elements yellow when focus is applied*/
input: focus{ background-color: yellow;}

\end{verbatim}


:link 和:visited 称为链接伪类,只能应用于锚元素。:hover、:active 和:focus 称为动态伪类,理论
上可以应用于任何元素。不幸的是,只有少数现代浏览器(比如 Firefox)支持这种功能。IE 6 和更低版本
只注意应用于锚链接的:active 和:hover 选择器,完全忽略: focus。
\subsubsection{通用选择器}
\label{sec-4.1.5}


\begin{verbatim}
*{ padding: 0; margin: 0; }

\end{verbatim}




\section{Question}
\label{sec-5}

\subsection{strong em}
\label{sec-5.1}

\subsection{blockquote cite}
\label{sec-5.2}

\subsection{abbr acronym code}
\label{sec-5.3}

\subsection{fieldset legend label}
\label{sec-5.4}

\subsection{caption thead tbody tfoot}
\label{sec-5.5}


\end{document}
